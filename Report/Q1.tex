\subsection{}
All data was collected from runs on a Hamilton par7.q node.

Figure [Fig.\ref{fig:scaling}] shows the scaling characteristics of step 4 in terms of the number of bodies. 
We see from [Fig.\ref{fig:scaling_sub}] that scenarios with over 1000 bodies scale well to Hamilton's 24 cores, whereas models using $<$1000 bodies scale less well. 
If we assume this is a strong scaling model, following $t(p) = f \cdot t(1) + (1-f)\cdot \frac{t(1)}{p}$, we can calculate constant $f$ for each data point [Fig.\ref{fig:scaling_f}]. We can see scenarios with $>=500$ bodies are modeled well by the strong scaling model as there is a low variance of $f$ estimations, giving evidence that $f$ is a constant. 
However, the strong scaling model breaks down under $500$ bodies as $f$ is likely not constant. 
I suspect this breakdown is caused by the overhead for OMP to create and synchronize threads. 
This overhead is likely a function of the number of threads and could explain the speedup then slowdown for $200$ bodies.
We also see that $f$ decreases with more bodies. In my implementation some of the calculations with runtime $O(n)$ are done in series (where $n$ is number of bodies); However, all calculations with runtime $O(n^2)$ are done in parallel. Hence we would expect f to decrease as $O(n^2)$ dominates.

\subsection{}
